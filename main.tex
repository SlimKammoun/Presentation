\documentclass[french]{beamer}
\usepackage[round]{natbib}

\usepackage{pgfpages}
%\setbeameroption{show notes}
%\setbeameroption{show notes on second screen=right}
\mode<presentation> {
  \usetheme{Madrid}
  % ou autre ...

  \setbeamercovered{transparent}
  % ou autre chose (il est également possible de supprimer cette ligne)
}
\newtheorem{proposition}[theorem]{Proposition}
\newtheorem{corollaire}[theorem]{Corollaire}

\usepackage[utf8]{inputenc}
\usepackage[T1]{fontenc}
\usepackage{babel}
\usepackage{times}
\usepackage[T1]{fontenc}
\usepackage{tikz}
\usepackage{amsfonts}
\usepackage{pgfplots}
\pgfplotsset{compat=newest}
\usepackage[vcentermath]{youngtab}
%\pgfdeclareimage[height=0.5cm]{le-logo}{logo-irisa}
%\logo{\pgfuseimage{le-logo}}
\setbeamertemplate{footline}[frame number]
\newtheorem{conjecture}[theorem]{Conjecture}

\usepackage{fourier, heuristica}
\usepackage{array, booktabs}
\usepackage{graphicx}
\usepackage[x11names,table]{xcolor}
\usepackage{caption}
\DeclareCaptionFont{blue}{\color{LightSteelBlue3}}

\newcommand{\foo}{\color{LightSteelBlue3}\makebox[0pt]{\textbullet}\hskip-0.5pt\vrule width 1pt\hspace{\labelsep}}


%%%%%%%%%%%%%%%%%%%%%%%%%%%
\title{Universalité pour les sous-suites croissantes des permutations aléatoires}

\subtitle {Forum Jeunes Mathématiciennes et Mathématiciens}
\author 
{ \large{Mohamed Slim Kammoun}
\\ \small{Université de Lille} \\ \ \\ \large{Avec Mylène Maida  et Adrien Hardy}}
\date { 29 November 2018}
\titlegraphic{
\includegraphics[height=0.9cm]{l1}
   \includegraphics[height=0.9cm]{0}
   \includegraphics[height=0.9cm]{l6}
   
}

\usepackage{xcolor}
\newcommand\ytl[2]{
\parbox[b]{5em}{\hfill{\color{cyan}\bfseries\sffamily #1}~$\cdots\cdots$~}\makebox[0pt][c]{$\bullet$}\vrule\quad \parbox[c]{5.5cm}{\vspace{7pt}\color{red!40!black!80}\raggedright\sffamily #2.\\[7pt]}\\[-3pt]}

\begin{document}

\begin{frame}
  \titlepage  
\end{frame}



\section*{Introduction}
\begin{frame}{Introduction}
\begin{itemize}
    \item Permutations aléatoires 
\begin{itemize}
    \item Distributions : loi uniforme, Ewens, Mallow, permutations virtuelles, Kingman etc.
    \item Objets combinatoires : nombre de cycles, longueurs des cycles, points fixes, suites monotone, les descentes etc.
    \item Applications : biologie (Ewens, mutations), combinatoire, processus coalescence, classification d etc. 
\end{itemize}
\item Universalité pour processus ponctuels (déterminantaux) 
\begin{itemize}
    \item Plus grande particule : Distribution de Tracy-Widom.
    \item Au bord : Processus de Airy
    \item Convergence globale : loi Semi-circulaire de Wigner
    \item Au "bulk" : Processus Sinus (discret) 
\end{itemize}

\end{itemize}
\end{frame}

\begin{frame}{Universalité}


\begin{table}
\small{}
\caption{Historique}
\centering
\begin{minipage}[t]{.7\linewidth}
\color{gray}
\rule{\linewidth}{1pt}
\ytl{1950's}{Travaux de Wigner}
\ytl{1963}{Wiger : Prix Nobel}
\ytl{1970}{Discussion Dyson & Montgomery : Zéro de $\zeta$ }
\ytl{1990's}{Matrices aléatoires, Universalité de Tao et Vu}
\ytl{1999}{Travaux de Baik, Deift et Johansson}
\ytl{2000's}{Modèles discrets}
\ytl{2010's}{Apllications : Télécom, statistiques, apprentissage automatique etc.}
\bigskip
\rule{\linewidth}{1pt}%
\end{minipage}%
\end{table}    
\end{frame}
\begin{frame}{Plan}
     \tableofcontents[
    hideothersubsections, 
    sectionstyle=show,
]
\end{frame}



\section{Objects combinatoires}
\begin{frame}{Plan}
\tableofcontents[currentsection,currentsubsection,
    hideothersubsections, 
    sectionstyle=show/shaded,
]
\end{frame}

\subsection{Plus longue sous-suite croissante}
\begin{frame}{Plus longue sous-suite croissante}
\begin{itemize}

\item $\mathfrak{S_n}$ : Le groupe symétrique d'ordre $n$ (l'ensemble des permutations de $\{1,2,\dots,n\}$).
\\ 
\item $i_1<i_2<\dots<i_k$ est une sous-suite croissante de longueur $k$ si $\sigma(i_1)<\sigma(i_2)\dots\sigma(i_k)$.

\item $\ell(\sigma)$  la longueur de la plus longue sous-suite croissante
\item Exemple \begin{equation*}\sigma=\begin{pmatrix}
 1& 2 & 3 & 4 & 5 \\ 
 5& 3 & 1 & 2 & 4 
\end{pmatrix}
\end{equation*}
$\ell(\sigma)=3$.

\end{itemize}
\end{frame}
\subsection{Correspondance de Robinson-Schinsted}

\begin{frame}{Tableaux de Young}
   \begin{definition}[Diagrammes de Young]
   $\lambda=(\lambda_i)_{i\geq1} \in \mathbb{N}^{\mathbb{N}^*}$ est un diagramme de Young de taille $n$ si 
   \begin{itemize}
       \item $\forall i\geq1, \  \lambda_{i+1}\leq \lambda_i$
       \item $\sum_{i=1}^\infty \lambda_i=n$
   \end{itemize}
   \end{definition}
   Exemple : les diagrammes de Young de taille 3 sont $\mathbb{Y}_3=(3,\underline{0}),(2,1,\underline{0}),(1,1,1,\underline{0})$
   \\   ou$\left(\yng(3),\yng(2,1),\yng(1,1,1)\right)$
   
 
\end{frame}
\begin{frame}{Tableau de Young}
    \begin{definition}[Tableau de Young]
    Un tableau de Young de forme $\lambda$ est un remplissage du diagramme de Young $\lambda$ avec les entrées $\{1,2,\dots,n\}$ strictement croissant sur chaque ligne et chaque colonne. 
    \end{definition}
    \begin{itemize}
        \item     Exemple : les tableaux de Young de forme $\yng(3,1)$ sont $\young(123,4),\young(124,3),\young(134,2)$. 
    \item$dim(\lambda)=$ \# tableau de Young de forme $\lambda$. 
    \item Exemple  $dim\left(\yng(3,1)\right)=3$.
    \item $dim(\lambda) = $   dimension de  la représentation du groupe symétrique indexé par $\lambda$.
    \end{itemize}
 \end{frame}


\begin{frame}{Correspondance de Robinson-Schinsted}
\begin{itemize}
    \item Bijection entre les permutations de $\mathfrak{S_n}$ et les paires de  tableaux de Young de même forme. 
    \item On note $\lambda(\sigma):=\lambda_i(\sigma)$ la forme de l'image de $\sigma$ par cette correspondance. 
    \item $\ell(\sigma)=\lambda_1(\sigma)$
\end{itemize}    
\end{frame}
\begin{frame}{Construction géométrique de Viennot}

\end{frame}


\section{Cas uniforme}
\begin{frame}{Plan}
\tableofcontents[currentsection,currentsubsection,
    hideothersubsections, 
    sectionstyle=show/shaded,
]
\end{frame}
\subsection{Conjecture d'Ulam}
\begin{frame}{Conjecture d'Ulam}
    \begin{conjecture}[\cite{ulam}]
    Si $\sigma_n \sim {U}_{\mathfrak{S}_n}$, alors
    $$\lim_{n\to \infty}\frac{\mathbb{E}(\ell(\sigma_n))}{\sqrt{n}}=c.$$ 
    \end{conjecture}
\begin{itemize}
    \item Si $\sigma_n \sim {U}_{\mathfrak{S}_n}$ alors 
    \begin{align*}
    \mathbb{P}(\lambda)&=\frac{\#\{\text{paires de tableaux de Young de forme } \lambda\}}{C}\\&=\frac{dim(\lambda)^2}{\sum_{\mu\in \mathbb{Y}_nμdim} dim( \mu)^2}=\frac{dim(\lambda)^2}{\# \mathfrak{S}_N}=\frac{dim(\lambda)^2}{n!} 
    \end{align*}
    \item Prouvé par \cite{vershik} ($c=2$).
\end{itemize}


\end{frame}

\subsection{Théorème de Baik, Deift et Johannson}
\begin{frame}{Théorème de Baik, Deift et Johannson}
    \begin{theorem}[\cite*{Baik1999}] 
Si $\sigma_n \sim U_{\mathfrak{S}_n}$ alors
\begin{equation*} 
\lim_{n \to \infty} \mathbb{P}\left(\frac{\ell(\sigma_n)-2\sqrt{n}}{n^\frac 16}\leq s\right)=F_2(s),
\end{equation*}
o\`u  $F_2$ est la distribution de Tracy-Widom.
\end{theorem}
On parle "d'universalité":
\begin{itemize}
    \item Plus grande (petite) valeurs propres de matrices aléatoires
    \item Modèles de particules comme le TASEP
\end{itemize}
\end{frame}

\subsection{Forme limite de Vershik-Kerov-Logan-Shepp}
\begin{frame}{Notations Russe}
\begin{itemize}
    \item On fait une rotation du tableau par $\frac{3\pi}{4}$.
    \item On regarde la fonction de hauteur complétée par $|x|$.
    \item On note la fonction obtenue $L_\lambda$.
\end{itemize}
\end{frame}

\begin{frame}{Notations Russes}
 \begin{figure}
\centering
\begin{tikzpicture}    [/pgfplots/y=0.42cm, /pgfplots/x=0.42cm]
      \begin{axis}[
    axis x line=center,
    axis y line=center,
    xmin=0, xmax=10,
    ymin=0, ymax=10, clip=false,
    ytick={0},
	xtick={0},
    minor xtick={0,1,2,3,3,4,5,6,7,8,9},
    minor ytick={0,1,2,3,3,4,5,6,7,8,9},
    grid=both,
    legend pos=north west,
    ymajorgrids=false,
    xmajorgrids=false, anchor=origin,
    grid style=dashed    , rotate around={45:(rel axis cs:0,0)}
,
]

\addplot[
    color=blue,
        line width=3pt,
    ]
    coordinates {
    (0,10)(0,7)(1,7)(1,5)(2,5)(2,2)(3,2)(3,1)(5,1)(5,0)(10,0)
    };
 
\end{axis}
\begin{axis}[
	axis x line=center,
    axis y line=center,
    xmin=-7.07, xmax=7.07,
    ymin=0, ymax=8, anchor=origin, clip=false,
    xtick={-7,-6,-5,-4,-3,-2,-1,0,1,2,3,4,5,6,7},
    ytick={0,1,2,3,3,4,5,6,7,8},
    legend pos=north west,
    ymajorgrids=false,
    xmajorgrids=false,rotate around={0:(rel axis cs:0,0)},
    grid style=dashed];
\end{axis}
    \end{tikzpicture}
    \caption{ $L_{(7,5,2,1,1,\underline{0})}$}
     \label{figL}
\end{figure} 
\end{frame}

\begin{frame}{Forme limite de Vershik-Kerov-Logan-Shepp}
    \begin{theorem}[\cite{vershik,LOGAN}]
    Si $\sigma_n \sim U_{\mathfrak{S}_n}$, alors pour tout $\varepsilon>0$,
\begin{align*}
\lim_{n\to \infty} \mathbb{P}\left(\sup_{s\in \mathbb{R}} \left|\frac{1}{\sqrt{2n}}L_{\lambda(\sigma_n)}\left({s}{\sqrt{2n}}\right)-\Omega(s)\right|<\varepsilon\right) =1,
\end{align*}
où
\begin{align*}
\Omega(s):=\begin{cases}
\frac{2}{\pi}(s\arcsin({s})+\sqrt{1-s^2}) & \text{ if } |s|<1 \\ 
|s| & \text{ if } |s|\geq 1 
\end{cases}.
\end{align*}
    \end{theorem}
    Il y a un lien entre $\Omega$ et la loi semi-circulaire de Wigner.
\end{frame}

\section{Généralisations}
\begin{frame}{Plan}
\tableofcontents[currentsection,currentsubsection,
    hideothersubsections, 
    sectionstyle=show/shaded,
]
\end{frame}
\subsection{Plus longue sous-suite croissante}
\begin{frame}{Plus longue sous-suite croissante}
    \begin{theorem}[\cite{sk}]
Supposons que la suite de permutations aléatoires  $(\sigma_n)_{n\geq 1}$ satisfait:
\begin{itemize}
\item  Pour tout entier strictement positif $n$ pour toutes permutations  $\sigma,\rho \in \mathfrak{S}_n$,
\begin{equation*}\label{h1}
\mathbb{P}(\sigma_n=\sigma)=\mathbb{P}(\sigma_n=\rho^{-1}\sigma\rho).
\end{equation*}
\item Pour tout $\varepsilon>0$,
\begin{equation*}\label{h2}
\lim_{n\to \infty}\mathbb{P}\left(\frac{\#(\sigma_n)}{n^\frac 16 }>\varepsilon\right) =0.
\end{equation*}
\end{itemize}
Alors pour tout  $s \in \mathbb{R}$,
\begin{equation*} \label{TW} 
\lim_{n\to \infty} \mathbb{P}\left(\frac{\ell(\sigma_n)-2\sqrt{n}}{n^\frac 16}\leq s\right)=\lim_{n\to \infty} \mathbb{P}\left(\frac{\underline{\ell}(\sigma_n)-2\sqrt{n}}{n^\frac 16}\leq s\right)=F_2(s).
\end{equation*}
\end{theorem}
\end{frame}
\subsection{Forme limite de Vershik-Kerov-Logan-Shepp}
 \begin{frame}{Forme limite}
    \begin{theorem}[\cite{sk}]
Supposons que la suite de permutations aléatoires  $(\sigma_n)_{n\geq 1}$ satisfait:
\begin{itemize}
\item  Pour tout entier strictement positif $n$ pour toutes permutations  $\sigma,\rho \in \mathfrak{S}_n$,
\begin{equation*}
\mathbb{P}(\sigma_n=\sigma)=\mathbb{P}(\sigma_n=\rho^{-1}\sigma\rho).
\end{equation*}
\item Pour tout $\varepsilon>0$,
\begin{equation*}
\lim_{n\to \infty}\mathbb{P}\left(\frac{\#(\sigma_n)}{n^\frac 12 }>\varepsilon\right) =0.
\end{equation*}
\end{itemize}
Alors pour  tout  $\varepsilon>0$,
\begin{equation*} 
\lim_{n\to \infty} \mathbb{P}\left(\frac{\ell(\sigma_n)-2\sqrt{n}}{n^\frac 16}\leq s\right)=\lim_{n\to \infty} \mathbb{P}\left(\frac{\underline{\ell}(\sigma_n)-2\sqrt{n}}{n^\frac 16}\leq s\right)=F_2(s).
\end{equation*}
\end{theorem}
\end{frame}
\subsection{Cas particulier: Ewens} 
\begin{frame}{Application : Ewens}
\begin{definition}[Loi d'Ewens]
Soit $\theta\geq 0$. On dit que $\sigma_n\sim Ew(\theta)$ si  
\begin{equation*}
\mathbb{P}(\sigma_n=\sigma)=\frac{\theta^{\#(\sigma)-1}}{\prod_{k=1}^{n-1}(\theta+k)}.\end{equation*}
\end{definition}
\begin{itemize}
\item $\theta$ Un taux de mutation.
\item  Chaque cycle représente un "type d'allèles". 
\item Structure logarithmique : $$\mathbb{E}(\#(\sigma_n))= 1+\sum_{k=1}^{n-1} \frac{\theta}{\theta+k}\sim \theta \log(n).$$
\end{itemize}
\end{frame}
\begin{frame}{Références}
\tiny

\bibliographystyle{abbrvnat}
    \bibliography{test}

\end{frame}
\begin{frame}

  \begin{columns}
    \begin{column}{0.70\textwidth}
      \begin{center}
        \includegraphics[width=\textwidth]{11}          
      \end{center}
    \end{column}
    \begin{column}{0.30\textwidth}
      \begin{center}

        \font\endfont = cmss10 at 5.40mm
\color[rgb]{0.00,0.00,1.00}       \endfont 
        \baselineskip 7.0mm

        Merci de votre attention

      \end{center}    

    \end{column}
  \end{columns}

\end{frame}
\end{document}